\documentclass[a4paper, 12pt]{article}
\usepackage[T1]{fontenc}
\usepackage[utf8]{inputenc}
\usepackage[italian]{babel}

%Risolve il problema di Adobe Reader che vedevo i caratteri dell'indice un po' grigi, non riuscendo a leggerli bene
\usepackage{lmodern}

\usepackage{natbib}
\usepackage{graphicx}

\usepackage{color}
\definecolor{myGreen}{rgb}{0,0.69,0.313}
\definecolor{myBlue}{rgb}{0.2157,0.3765,0.5725}
\definecolor{myBrown}{rgb}{0.59215,0.28235,0.02353}

\usepackage{listings}
\lstdefinelanguage{SQL}
{ keywords={CREATE, SCHEMA, TABLE, AUTHORIZATION, REFERENCES, UNIQUE, PRIMARY, KEY, FOREIGN, DOMAIN, DELETE, ON, UPDATE, DEFAULT, CONSTRAINT, NOT, NULL, CHECK, AND, OR, CASCADE, SELECT, FROM, WHERE},
	keywordstyle={\bfseries},
	columns=fixed,
	sensitive=false,
	showtabs=false,
	showspaces=false,
	tabsize=4,
	captionpos=b,
	basicstyle={\sffamily}
}
%Tabelle
\usepackage{tabularx}
\usepackage{booktabs}

%Serve per avere l'indice cliccabile. Con questi parametri inoltre non spuntano i riquadri rossi di default
%TODO Non dimenticare di togliere il commento su hyperref quando finisco la relazione
%\usepackage[colorlinks=true, linkcolor=black, citecolor=black, urlcolor=black]{hyperref}

% Impostazioni di pagina e margini
\usepackage[a4paper, margin=2.54cm]{geometry}

%Header and Footer
\usepackage{fancyhdr}
\pagestyle{fancy}
\fancyhf{}
\lhead{TIW - Gestione Preventivi - a.a. 2021/2022}
\cfoot{\thepage}
%warning per 12pt del carattere
\setlength{\headheight}{14.49998pt}
\addtolength{\topmargin}{-2.49998pt}
% Titolo e informazioni
\title{TIW - Gestione Preventivi}
\author{Riccardo Inghilleri - Matricola n. 937011\\Manuela Merlo - Matricola n. 936925}
\date{Anno Accademico 2021/2022}

\begin{document}
\begin{titlepage}
	\begin{center}
		\vspace*{1cm}
		
		\Huge
		TIW - Gestione preventivi\\
		\vspace{4cm}
		
		\includegraphics[width=0.4\textwidth]{PureHTML_images/polimilogo}
		
		\vspace{3.5cm}
		\LARGE
		Politecnico di Milano\\
		Anno Accademico 2021/2022\\
		\vspace{0.5cm}
		\Large
		Prof. Piero Fraternali\\
		
		\vspace{5cm}
		
		{Riccardo Inghilleri (Codice Persona 10713236 - Matricola 937011)\\Manuela Merlo (Codice Persona 10670533 - Matricola 936925)}
		
		\vfill
		
		\vspace{0.8cm}
		
	\end{center}
\end{titlepage}
\tableofcontents
\newpage
\section{Gestione Preventivi - Pure HTML}
\subsection{Analisi dei dati per il database}
Un’applicazione web consente la gestione di richieste di preventivi per prodotti personalizzati. L’applicazione supporta registrazione e login di \textbf{\textcolor{red}{clienti}} e \textbf{\textcolor{red}{impiegati}} mediante una pagina pubblica con opportune form. La registrazione controlla l’unicità dello \textbf{\textcolor{myGreen}{username}}. Un \textbf{\textcolor{red}{preventivo}} \textbf{\textcolor{myBlue}{è associato a un}} \textbf{\textcolor{red}{prodotto}}, \textbf{\textcolor{myBlue}{al cliente che l’ha richiesto}} e \textbf{\textcolor{myBlue}{all’impiegato che l’ha gestito}}. Il preventivo \textbf{\textcolor{myBlue}{comprende una o più} \textcolor{red}{opzioni} \textcolor{myBlue}{per il prodotto a cui è associato}}, che devono essere tra quelle disponibili per il prodotto. Un prodotto ha un \textbf{\textcolor{myGreen}{codice}}, un’\textbf{\textcolor{myGreen}{immagine}} e un \textbf{\textcolor{myGreen}{nome}}. Un’opzione ha un \textbf{\textcolor{myGreen}{codice}}, un \textbf{\textcolor{myGreen}{tipo}} (“normale”, “in offerta”) e un \textbf{\textcolor{myGreen}{nome}}. Un preventivo ha un \textbf{\textcolor{myGreen}{prezzo}}, definito dall’impiegato. Quando l’utente (cliente o impiegato) accede all’applicazione, appare una LOGIN PAGE, mediante la quale l’utente si autentica con username e \textbf{\textcolor{myGreen}{password}}. Quando un cliente fa login, accede a una pagina HOME PAGE CLIENTE che contiene una form per creare un preventivo e l’elenco dei preventivi creati dal cliente. Selezionando uno dei preventivi il cliente ne visualizza i dettagli. Mediante la form di creazione di un preventivo l’utente per prima cosa sceglie il prodotto; scelto il prodotto, la form mostra le opzioni di quel prodotto. L’utente sceglie le opzioni (almeno una) e conferma l’invio del preventivo mediante il bottone INVIA PREVENTIVO. Quando un impiegato effettua il login, accede a una pagina HOME PAGE IMPIEGATO che contiene l’elenco dei preventivi gestiti da lui in precedenza e quello dei preventivi non ancora associati a nessun impiegato. Quando l’impiegato seleziona un elemento dall’elenco dei preventivi non ancora associati a nessuno, compare una pagina PREZZA PREVENTIVO che mostra i dati del cliente (username) e del preventivo e una form per inserire il prezzo del preventivo. Quando l’impiegato inserisce il prezzo e invia i dati con il bottone INVIA PREZZO, compare di nuovo la pagina HOME PAGE IMPIEGATO con gli elenchi dei preventivi aggiornati. Il prezzo definito dall’impiegato risulta visibile al cliente quando questi accede all’elenco dei propri preventivi e visualizza i dettagli del preventivo. La pagina PREZZA PREVENTIVO contiene anche un collegamento per tornare alla HOME PAGE IMPIEGATO. L’applicazione consente il logout dell’utente.\\

\noindent \textbf{\textcolor{red}{Entities}, \textcolor{myGreen}{attributes}, \textcolor{myBlue}{relationships}}
\newpage
\subsection{Database Design}
\begin{figure}[h!]
	\centering
	\includegraphics[width=1\textwidth]{PureHTML_images/quotemanagementdesign1.png}
	\caption{Database Desing}
	\label{figure:database_design}
\end{figure}
\subsection{Local Database schema}
\begin{lstlisting}[language=SQL]
CREATE TABLE `user` (
`id` int AUTO_INCREMENT,
`username` varchar(45),
`email` varchar(255),
`password` varchar(45) NOT NULL,
`role` varchar(45) NOT NULL,
PRIMARY KEY (`id`),
UNIQUE(`email`),
CONSTRAINT `UC_UsernameRole` UNIQUE (`username`,`role`),
CONSTRAINT `NN_UsernameORRole` 
CHECK (`username` is not null OR `email` is not null))

CREATE TABLE `product` (
`code` int AUTO_INCREMENT,
`image` varchar(45) NOT NULL,
`name` varchar(45) NOT NULL,
PRIMARY KEY (`code`))


CREATE TABLE `option` (
`code` int AUTO_INCREMENT,
`type` varchar(45) NOT NULL,
`name` varchar(45) NOT NULL,
PRIMARY KEY (`code`))


CREATE TABLE `quote` (
`id` int auto_increment,
`clientId` int NOT NULL,
`workerId` int,
`productCode` int NOT NULL,
`price` int default null,
PRIMARY KEY (`id`),
CONSTRAINT `asstoclient` FOREIGN KEY (`clientId`) 
REFERENCES `client` (`id`) ON DELETE CASCADE,
CONSTRAINT `asstoworker` FOREIGN KEY (`workerId`) 
REFERENCES `worker` (`id`) ON DELETE CASCADE,
CONSTRAINT `asstoproduct` FOREIGN KEY (`productCode`) 
REFERENCES `product` (`code`) ON DELETE CASCADE)


CREATE TABLE `productoptions`(
`productCode` int,
`optionCode` int,
PRIMARY KEY (`productCode`,`optionCode`),
CONSTRAINT `relwithproduct` FOREIGN KEY (`productCode`) 
REFERENCES `product` (`code`) ON DELETE CASCADE,
CONSTRAINT `relwithoption` FOREIGN KEY (`optionCode`) 
REFERENCES `option` (`code`) ON DELETE CASCADE)


CREATE TABLE `quoteoptions`(
`quoteId` int NOT NULL,
`optionCode` int NOT NULL,
PRIMARY KEY (`quoteId`,`optionCode`),
CONSTRAINT `relwithquote` FOREIGN KEY (`quoteId`) 
REFERENCES `quote` (`id`) ON DELETE CASCADE,
CONSTRAINT `selectedoption` FOREIGN KEY (`optionCode`) 
REFERENCES `option` (`code`) ON DELETE CASCADE)
\end{lstlisting}
\subsection{Analisi dei dati per i requisiti dell'applicazione}
Un’applicazione web consente la gestione di richieste di preventivi per prodotti personalizzati. L’applicazione supporta \textbf{\textcolor{myBrown}{registrazione}} e \textbf{\textcolor{myBrown}{login}} di clienti e impiegati mediante una pagina pubblica con opportune \textbf{\textcolor{myGreen}{form}}. La registrazione controlla l’unicità dello username. Un preventivo è associato a un prodotto, al cliente che l’ha richiesto e all’impiegato che l’ha gestito. Il preventivo comprende una o più opzioni per il prodotto a cui è associato, che devono essere tra quelle disponibili per il prodotto. Un prodotto ha un codice, un’immagine e un nome. Un’opzione ha un codice, un tipo (“normale”, “in offerta”) e un nome. Un preventivo ha un prezzo, definito dall’impiegato. Quando l’utente (cliente o impiegato) \textbf{\textcolor{myBlue}{accede all’applicazione}}, \textbf{\textcolor{myBrown}{appare}} una \textbf{\textcolor{red}{LOGIN PAGE}}, mediante la quale \textbf{\textcolor{myBlue}{l’utente si autentica}} con username e password. Quando un cliente fa login, accede a una pagina \textbf{\textcolor{red}{HOME PAGE CLIENTE}} che contiene una \textbf{\textcolor{myGreen}{form}} \textbf{\textcolor{myBrown}{per creare un preventivo}} e \textbf{\textcolor{myGreen}{l’elenco dei preventivi}} creati dal cliente. \textbf{\textcolor{myBlue}{Selezionando uno dei preventivi}} il cliente ne \textbf{\textcolor{myGreen}{visualizza i dettagli}}. Mediante la form di creazione di un preventivo l’utente per prima cosa sceglie il prodotto; \textbf{\textcolor{myBlue}{scelto il prodotto}}, la form mostra \textbf{\textcolor{myGreen}{le opzioni di quel prodotto}}. L’utente sceglie le opzioni (almeno una) e \textbf{\textcolor{myBlue}{conferma l’invio}} del preventivo mediante il \textbf{\textcolor{myGreen}{bottone INVIA PREVENTIVO}}. Quando un impiegato effettua il login, accede a una pagina \textbf{\textcolor{red}{HOME PAGE IMPIEGATO}} che contiene l’\textbf{\textcolor{myGreen}{elenco dei preventivi}} gestiti da lui in precedenza e \textbf{\textcolor{myGreen}{quello dei preventivi non ancora associati}} a nessun impiegato. Quando l’impiegato \textbf{\textcolor{myBlue}{seleziona un elemento}} dall’elenco dei preventivi non ancora associati a nessuno, compare una \textbf{\textcolor{red}{pagina PREZZA PREVENTIVO}} che mostra \textbf{\textcolor{myGreen}{i dati del cliente (username) e del preventivo}} e una \textbf{\textcolor{myGreen}{form}} \textbf{\textcolor{myBrown}{per inserire il prezzo del preventivo}}. Quando l’impiegato \textbf{\textcolor{myBlue}{inserisce il prezzo e invia i dati}} con il \textbf{\textcolor{myGreen}{bottone INVIA PREZZO}}, compare di nuovo la pagina HOME PAGE IMPIEGATO con gli elenchi dei preventivi aggiornati. Il prezzo definito dall’impiegato risulta visibile al cliente quando questi \textbf{\textcolor{myBlue}{accede all’elenco dei propri preventivi}} e visualizza i dettagli del preventivo. La pagina PREZZA PREVENTIVO contiene anche un collegamento per tornare alla HOME PAGE IMPIEGATO. L’applicazione consente il \textbf{\textcolor{myBrown}{logout}} dell’utente.\\

\noindent \textbf{\textcolor{red}{Pages(views)}, \textcolor{myGreen}{views components}, \textcolor{myBlue}{events}, \textcolor{myBrown}{actions}}
\subsection{Completamento delle specifiche}
\begin{itemize}
	\item La pagina di default contiene la form di login;
	\item Lo username e la password non possono essere nulli;
	\item Un client e un worker possono avere lo stesso username, due client o due worker invece no;
	\item Il prezzo non può essere negativo;
	\item \'E permesso al client di richiedere preventivi dello stesso prodotto con le stesse opzioni;
	\item Nel caso di riapertura del sito da parte di un utente che precedentemente non ha effettuato il logout, non viene richiesta una nuova autenticazione e viene aperta direttamente la sua home page.
 
\end{itemize}
\newpage
\subsection{Application Desing}
\begin{figure}[h!]
	\centering
	\includegraphics[width=0.97\textwidth]{PureHTML_images/ifml.png}
	\caption{IFML diagram}
	\label{figure:ifml}
\end{figure}
\newpage
\subsection{Components}

\begin{itemize}
\item \textbf{Model Objects (Beans)}
\begin{itemize}
\item Option
\item Product
\item Quote
\item User
\end{itemize}
\item \textbf{Data Access Objects (Classes)}
\begin{itemize}
	\item \textbf{OptionDAO}
	\begin{itemize}
		\item boolean hasOptionByCode(int productCode, int optionCode);
		\item List<Option> findOptionsByProductCode(int productCode);
		\item List<Option> findOptionsByQuoteId(int quoteId);
		\item void insertOption(int quoteId, int optionCode);
	\end{itemize}
	\item \textbf{ProductDAO}
	\begin{itemize}
		\item List<Product> findAllProducts();
		\item Product findProductByCode(int code);
	\end{itemize}
	\item \textbf{QuoteDAO}
	\begin{itemize}
		\item List<Quote> findQuotesByUserId(int userId, String role);
		\item Quote findQuoteById(int quoteId);
		\item List<Quote> findUnmanagedQuotes();
		\item int insertQuote(int clientId, int productCode);
		\item void updateQuote(int quoteId, int workerId, int price);
	\end{itemize}
	\item \textbf{UserDAO}
	\begin{itemize}
		\item User findUser(String username, String password, String role);
		\item User findUser(String username, String role);
		\item User findClientById(int clientId);
		\item void registerUser(String username, String password, String role);
	\end{itemize}
\end{itemize}
\item \textbf{Controllers (Servlets)}
\begin{itemize}
	\item GotoLogin
	\item CheckLogin
	\item Register
	\item GotoClientHome
	\item GotoWorkerHome
	\item CreateQuote
	\item GotoQuoteDetails
	\item UpdatePrice
	\item Logout
\end{itemize}
\item \textbf{Filters}
\begin{itemize}
	\item SessionChecker
	\item ClientChecker
	\item WorkerChecker
\end{itemize}
\item \textbf{Views (Templates)}
\begin{itemize}
\item Login.html
\item Register.html
\item ClientHome.html
\item WorkerHome.html
\item QuoteDetails.html
\end{itemize}
\end{itemize}

\subsection{Events}
I filtri utilizzati sono i seguenti:\\

\noindent \textbf{SessionChecker:}
\begin{itemize}
\item \textbf{Controllo:} session.isNew() || session.getAttribute("currentUser") == null 
\item \textbf{Azione:} redirect alla Login Page
\item \textbf{Quando:} prima di ogni Servlet
\end{itemize}
\textbf{ClientChecker:} controlla che lo user corrente sia un Client
\begin{itemize}
\item \textbf{Controllo:} !user.getRole().equals("client")
\item \textbf{Azione:} redirect alla Login Page
\item \textbf{Quando:} prima di /GotoClientHome e /CreateQuote
\end{itemize}
\textbf{WorkerChecker:} controlla che lo user corrente sia un Worker
\begin{itemize}
\item \textbf{Controllo:} !user.getRole().equals("worker")
\item \textbf{Azione:} redirect alla Login Page
\item \textbf{Quando:} prima di /GotoWorkerHome e /UpdatePrice
\end{itemize}
\newpage
\subsubsection{Go to Login}
\begin{figure}[h!]
	\centering
	\includegraphics[width=1\textwidth]{PureHTML_images/GotoLogin.png}
	\caption{Event - Go to Login}
	\label{figure:gotologin_sd}
\end{figure}
\subsubsection{Login}
\begin{figure}[h!]
	\centering
	\includegraphics[width=1\textwidth]{PureHTML_images/Login.png}
	\caption{Event - Login}
	\label{figure:login_sd}
\end{figure}
\noindent \textbf{Controlli:}

\noindent \textbf{HTML:} \\ Lo username e la password sono obbligatori.

\noindent \textbf{Server:} 

\noindent Viene controllato che i parametri della Form non sia nulli, vuoti o invalidi:

\begin{lstlisting}[language=java] 
(username = null || role = null || 
(!role.equalsIgnoreCase("client") && !role.equalsIgnoreCase("worker")) 
|| password = null || 
username.isEmpty() || password.isEmpty())
\end{lstlisting}

\subsubsection{Register}
\begin{figure}[h!]
	\centering
	\includegraphics[width=1\textwidth]{PureHTML_images/Register.png}
	\caption{Event - Register}
	\label{figure:register_sd}
\end{figure}
\noindent \textbf{Controlli:}\\
\noindent \textbf{Server:} 
\noindent Viene controllato che :
\begin{enumerate}
\item I parametri della form non siano nulli o vuoti ( username - password - ruolo ).
\item Che all'interno del DB non esista un altro utente con lo stesso ruolo, avente lo stesso username.
\end{enumerate}



\subsubsection{GotoClientHome}
\begin{figure}[h!]
	\centering
	\includegraphics[width=1\textwidth]{PureHTML_images/GotoClientHome.png}
	\caption{Event - Go to Client Home}
	\label{figure:gotoclienthome_sd}
\end{figure}
\noindent \textbf{Controlli:}\\
\noindent \textbf{Server:} 
\noindent Al ricaricamento della pagina, dopo la selezione del prodotto per la richiesta di un nuovo preventivo, viene controllato che :
\begin{enumerate}
\item L'id del prodotto, se selezionato, sia in un formato numerico e corrisponda ad un id effetivamente presente nel DB.
\end{enumerate}
\newpage
\subsubsection{GotoWorkerHome}
\begin{figure}[h!]
	\centering
	\includegraphics[width=1\textwidth]{PureHTML_images/GotoWorkerHome.png}
	\caption{Event - Go to Worker Home}
	\label{figure:gotoworkerhome_sd}
\end{figure}
\noindent \textbf{Controlli:}\\
\noindent \textbf{Server:} 
\noindent Dato che la raccolta dei dati necessari per la visualizzazione della pagina non dipendono da nessun parametro di input, ma solo dall'id dello user presente nella sessione, non vengono effettuati controlli aggiuntivi a quelli 
già fatti dai filtri ovvero: che un utente sia effettivamente loggato e che quest'ultimo sia effetivamente un worker.
\newpage
\subsubsection{GotoQuoteDetails}
\begin{figure}[h!]
	\centering
	\includegraphics[width=1\textwidth]{PureHTML_images/GotoQuoteDetails.png}
	\caption{Event - Go to Quote Details}
	\label{figure:gotoquotedetails_sd}
\end{figure}
\noindent \textbf{Controlli:}\\
\noindent \textbf{Server:} 
\noindent Viene controllato che :
\begin{enumerate}
\item L'id del prodotto non sia nullo, vuoto, sia in un formato numerico e che corrisponda ad un id effetivamente presente nel DB.
\item Che lo user presente nella sessione abbia l'autorizzazione per visualizzare i dettagli del preventivo.
\end{enumerate}
\newpage
\subsubsection{CreateQuote}
\begin{figure}[h!]
	\centering
	\includegraphics[width=1\textwidth]{PureHTML_images/CreateQuote1.png}
	\label{figure:createquote1_sd}
\end{figure}
\newpage
\begin{figure}[h!]
	\centering
	\includegraphics[width=1\textwidth]{PureHTML_images/CreateQuote2.png}
	\caption{Event - Create Quote}
	\label{figure:createquote2_sd}
\end{figure}
\noindent \textbf{Controlli:}

\noindent \textbf{HTML:} \\ La scelta di almeno un'opzione per la creazione del preventivo è obbligatoria.

\noindent \textbf{Server:} 

\noindent Viene controllato che :
\begin{enumerate}
\item L'id del prodotto non sia nullo, vuoto, sia in un formato numerico e che corrisponda ad un id effetivamente presente nel DB.
\item Il parametro \verb|chosenOption| non sia nullo o vuoto. Inoltre viene controllato che ogni elemento dell'array non sia nullo, vuoto , sia in formato numerico e corrisponda effetivamente ad una opzione
disponibile per il prodotto selezionato.
\end{enumerate}
\newpage
\subsubsection{UpdatePrice}
\begin{figure}[h!]
	\centering
	\includegraphics[width=1\textwidth]{PureHTML_images/UpdatePrice.png}
	\caption{Event - Update Price}
	\label{figure:updateprice_sd}
\end{figure}
\noindent \textbf{Controlli:}\\
\noindent \textbf{Server:} 
\noindent Viene controllato che :
\begin{enumerate}
\item L'id del prodotto non sia nullo, vuoto, sia in un formato numerico e che corrisponda ad un id effetivamente presente nel DB.
\item Che il preventivo non sia già stato prezzato da un altro worker.
\item Che il prezzo inserito non sia nullo,vuoto, sia in un formato numerico e maggiore di 0.
\end{enumerate}

\newpage
\section{Gestione Preventivi - RIA}

\end{document}